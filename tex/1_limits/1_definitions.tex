\section{Definitions of Limits}

\begin{defn}{Limit of a Function (Early Definition)}
Suppose the function $f$ is defined for all $x$ near some value $a$ except possibly at $a$ itself. If $f(x)$ is arbitrarily close to some value $L$ whenever $x$ is sufficiently close to $a$ (without actually equaling it), then we write
\[\lim_{x\to a} f(x) = L\]
\end{defn}

\begin{note}{Quick Notes}
\hspace{1cm}
\begin{itemize}
\item Sometimes we'll write $f(x) \to L$ as $x\to a$.
\item Limits do not have anything to do with function values, since it depends on values \textit{near} $a$.
\item We can make $x$ ``get close to'' $a$ in two ways (from the left or from the right).
\item We can investigate limits in three ways: graphically, numerically, and analytically.
\end{itemize}
\end{note}

\begin{defn}{Graphically}
As you ``move'' closer to the $x$-value on the graph that you care about, what is happening to the $y$-values? Notice that since we only ``get close'' to $a$, we don't care about the specific function value.
\end{defn}

\begin{defn}{Numerically}
Pick $x$-values that are close to $a$ and create a table of values. What is the pattern?
\end{defn}

\section*{One-Sided Limits}

\begin{defn}{Right-Sided Limit}
If $f$ is close to $L$ whenever $x$ is close to, but greater than, $a$, then we write:
\[ \lim_{x\to a^+} f(x)=L\]
\end{defn}

\begin{defn}{Left-Sided Limit}
If $f$ is close to $L$ whenever $x$ is close to, but less than, $a$, then we write:
\[ \lim_{x\to a^-} f(x)=L\]
\end{defn}

\begin{thm}{Relationship Between One and Two-Sided Limits}
Assume $f$ is defined for all $x$ near $a$ except possibly at $a$. Then $\dlim_{x\to a} f(x) = L$ if and only if $\dlim_{x\to a^-}f(x) = L = \dlim_{x\to a^+} f(x)$.
\end{thm}

\section*{Examples}

\begin{itemize}
\item Graph a small piecewise function with a hole, but the limit existing.
\item Graph a small piecewise function where the left side limit does not equal the right sided limit.
\item Numerically find:
\begin{multicols}{3}
\begin{itemize}
  \item $\dlim_{x\to 4} \dfrac{x-4}{\sqrt{x}-2}$
  \item $\dlim_{x\to 1} \dfrac{\sqrt{x}-1}{x-1}$
  \item $\dlim_{x\to 2} \dfrac{x^3-8}{4(x-2)}$
\end{itemize}
\end{multicols}
\item Examine $\dlim_{x\to 0} \cos\left(\dfrac{1}{x}\right)$. Numerically, it looks like $\dlim_{x\to 0} \cos\left(\dfrac{1}{x}\right)=-1$, but it actually oscillates between $-1$ and 1.\\
Note that when $x=\dfrac{1}{n\pi}$, we can see that  $\cos\left(\dfrac{1}{x}\right)=\cos(n\pi)$. For bigger values of $n$, we know that $x=\dfrac{1}{n\pi} \to 0$, and we can also use some basic trig knowledge to analyze what happens when $n$ is odd and when $n$ is even. So as $n$ gets bigger, it bounces back and forth between even and odd values, meaning the value of $\cos(n\pi)$ oscillates between $-1$ and $1$.\\

So, show that $\dlim_{x\to0^+}\cos(1/x)$ does not exist to show that the two-sided limit does not exist.
\end{itemize}
