\section{Definitions of Limits}

We begin with the definition of the limit of a function.

\begin{thm}{Limit of a Function}
  Consider the function $f$, defined at all $x$ around some value $a$ except possibly at $a$ itself.
  If $f(x)$ is arbitrarily close to the single real number $L$ whenever $x$ is sufficiently close to, but not equal to, $a$, then we say that the limit of $f$ is $L$ as $x$ approaches $a$, and we write
  \[\lim_{x\to a} f(x) = L\]
  Alternatively, we can write that $f(x)\to L$ when $x\to a$.
\end{thm}

We note that in this definition, we do not actually consider $f(a)$.
In fact, the function value at $x=a$ doesn't even have to exist.
We only care about the function values \textit{around} the point at $x=a$.

When we investigate limits, we'll do so in three different ways:\footnote{
These three methods of investigation will remain consistent for most new concepts we discuss in this text.}
\begin{enumerate}
  \item Numerically
  \item Graphically
  \item Analytically
\end{enumerate}

In this section, we will discuss some numerical and graphical approximations of limits, and will build up to analytical evaluations of a limit in the next few sections.

\section*{Numerical Approximations}

Consider the function $f(x) = x-4$.

Let's say we want to investigate the behavior of the function\footnote{We're mainly interested in whether or not there's a specific $y$-value that the function is arbitrarily close to.} \textit{around} $x=2$.
What are the function values doing around that point?
When $x$ is sufficiently close to $2$, is $f(x)$ arbitrarily close to some single real number?

If I were to ask you these questions, how would you approach answering them?

Let's start by just picking some $x$-values that are ``close to'' $2$.
What $x$-values are you thinking of?

Let's start with $x=1.5$.
We can find $f(1.5)$ pretty easily:
\begin{align*}
  f(1.5) & = 1.5-4\\
  & = -2.5
\end{align*}

Our predicament now is that we are trying to see what $y$-value (if any), this is arbitrarily close to.
We don't really hanve enough evidence to say right now.
Maybe we need an $x$-value that is \textit{closer} to $2$.

Let's try $x=1.9$.
\begin{align*}
  f(1.9) & = 1.9-4\\
  & = -2.1
\end{align*}

We might have an idea now of what $f(x)$ is close to.
We can always try this again, maybe with $x=1.99$ now.
\begin{align*}
  f(1.99) & = 1.99-4\\
  & = -2.01
\end{align*}

We can keep this going a bit more:

\begin{center}
  \begin{tabular}{cccccc} \toprule
    $\bm{x}$ & 1.5 & 1.9 & 1.99 & 1.999 & 1.9999\\ \midrule
    $\bm{f(x)}$ & $-2.5$ & $-2.1$ & $-2.01$ & $-2.001$ & $-2.0001$\\ \bottomrule
  \end{tabular}
\end{center}

So, do we have a good idea of what real number $f(x)$ is getting arbitrarily close to when $x$ is sufficiently close to $2$?\footnote{Here's something to think about: why not just start with $x=2$? Go back to the definition of a limit and see.}

It should be obvious in this case: it looks like $f(x) \to -2$ when $x\to 2$.
We have some good evidence that $\dlim_{x\to 2}x-4 = -2$.

But wait, is this ok? We've tried to look at $x$-values that were sufficiently close to $2$, but we've only looked at $x$-values that are \textit{less than} 2.
Why not repeat this process with $x$-values that are close to, but slightly larger than, 2?

\begin{center}
  \begin{tabular}{cccccc} \toprule
    2.0001 & 2.001 & 2.01 & 2.1 & 2.5 & $\bm{x}$\\ \midrule
    $-1.9999$ & $-1.999$ & $-1.99$ & $-1.9$ & $-1.5$ & $\bm{f(x)}$ \\ \bottomrule
  \end{tabular}
\end{center}

Well this is good news! It still looks like $f(x) \to -2$ when $x\to 2$.
We probably have enough evidence to believe that $\dlim_{x\to 2}(x-4) = -2$.

Maybe more importantly, though, we've touched on a very important concept with limits.

\subsection*{One-Sided Limits}

We've seen already, then, that there are two ``sides'' to a limit, since we can look at values of $x$ that are \textit{close to} $a$ on either side: $x$-values that less than $a$, and $x$-values that are greater than $a$.\footnote{We use the terminology ``left'' and ``right'' because of the number line: values less than $a$ on a number line are on the left of $a$, and values greater than $a$ on the number line are on the right of $a$.}

\begin{defn}{Right-Sided Limit}
  If $f$ is arbitrarily close to the single real-number $L$ whenever $x$ is sufficiently close to, but greater than, $a$, then we write:
  \[ \lim_{x\to a^+} f(x)=L\]
\end{defn}

\begin{defn}{Left-Sided Limit}
  If $f$ is arbitrarily close to the single real-number $L$ whenever $x$ is sufficiently close to, but less than, $a$, then we write:
  \[ \lim_{x\to a^-} f(x)=L\]
\end{defn}

In our example above, whether we looked at $x$-values that were slightly less than $2$ or $x$-values that were slightly greater than $2$, the function values seemed to be getting arbitrarily close to $-2$.
Since both the right-sided limit and the left-sided limit were the same, we decided that we had more than enough evidence to conclude that the limit of the function as $x$ approached $2$ was $-2$.

What would we have concluded if the one-sided limits did not match?

\begin{thm}{The Existence (and Non-Existence) of a Limit}
  If $f$ is some function defined at all $x$-values on either side of $a$ except possibly at $a$ itself and $L$ is some real number, then we say that $\dlim_{x\to a} f(x) = L$ if and only if $\dlim_{x\to a^-}f(x) = L$ and $\dlim_{x\to a^+} f(x) = L$.

  If $\dlim_{x\to a^-} f(x) \neq \dlim_{x\to a^+} f(x)$, then we say that $\dlim_{x\to a} f(x)$ does not exist.
\end{thm}

\subsection*{Another Example}

Let's approximate $\dlim_{x\to -3} x^2+1$ numerically.

\textbf{Left-Sided Limit}

\begin{tabular}{ccccc} \toprule
  $\bm{x}$ & $-3.5$ & $-3.1$ & $-3.01$ & $-3.001$ \\ \midrule
  $\bm{f(x)}$ & $13.25$ & $10.61$ & $10.0601$ & $10.006001$\\ \bottomrule
\end{tabular}

\begin{flushright}
  \textbf{Right-Sided Limit}

  \begin{tabular}{ccccc} \toprule
    $-2.999$ & $-2.99$ & $-2.9$ & $-2.5$ & $\bm{x}$ \\ \midrule
    $9.994001$ & $9.9401$ & $9.41$ & $7.25$ & $\bm{f(x)}$ \\ \bottomrule
  \end{tabular}
\end{flushright}

Likely this enough for us to believe that $f(x) \to 10$ when $x\to -3$, or $\dlim_{x\to-3} (x^2+1)=10$.

We'll see some more examples of numerical approximation near the end of this section, but first let's look at things graphically.

\section*{Graphical Approximations}

With graphical approximations, we're essentially going to repeat the process that we've done for numerical approximations.
The main difference is in the representation of the function.

When we had the function represented in some algebraic expression ($f(x) = ...$), we could approximate the limit $\dlim_{x\to a} f(x)$ by evaluating our function at $x$-values.
We picked $x$-values that were ``sufficiently close'' to $a$ on either side, and tried to figure out what the corresponding $y$-values were ``arbitrarily close'' to.

In this context, though, we'll have functions represented graphically.
We still want to evaluate our functions at a bunch of $x$-values and find the corresponding $y$-values.
But now, instead of plugging those $x$-values into an algebraic rule or expression, we'll be looking at a picture of a graph to find those $y$-values.

Consider this function $f(x)$, graphed below.

\begin{figure}[h!tb]
  \includegraphics[scale=0.75]{./1_limits/images/1-1_graph1.png}
  \centering
\end{figure}

If we look at the limit $\dlim_{x\to 3} f(x)$, we can try to evaluate our function near $3$.
Instead of actually evaluating it, though, we'll have to look at the $y$-values from the graph.

\textbf{Left-Sided Limit}

\begin{figure}[h!tb]
  \includegraphics[scale=0.75]{./1_limits/images/1-1_graph1L.png}
  \centering
\end{figure}

Graphically, then, it looks like as $x\to 3^-$, $f(x) \to -2$.

\textbf{Right-Sided Limit}

\begin{figure}[h!tb]
  \includegraphics[scale=0.75]{./1_limits/images/1-1_graph1R.png}
  \centering
\end{figure}

And here, based on the graph, it looks like as $x\to 3^+$, $f(x)\to -2$.

So we have graphical evidence that $\dlim_{x\to 3^-} f(x) = \dlim_{x\to 3^+} f(x) = -2$.

Graphically, we can quickly approximate limits relatively easily.
With this visual medium, we can get a ton of information quickly just by glancing at our graph.
We can get approximations on both of the one-sided limits.

\begin{figure}[h!tb]
  \includegraphics[scale=0.75]{./1_limits/images/1-1_graph1bar1.png}
  \centering
\end{figure}

\begin{figure}[h!tb]
  \includegraphics[scale=0.75]{./1_limits/images/1-1_graph1bar2.png}
  \centering
\end{figure}

Let's use this method to approximate another limit.

% Second graph.

% Second graph with bar.

Visually, it looks like $\dlim_{x\to b^-} g(x) = L_1$, but $\dlim_{x\to b^+} g(x) = L_2$. Thus, since we have evidence that $\dlim_{x\to b^-} g(x) \neq \dlim_{x\to b^+} g(x)$, we should probably conclude that the limit $\dlim_{x\to b} g(x)$ does not exist.

\section*{A Warning}

Hopefully we can see how these numerical and graphical approximations are helpful in building some intuition about whether a limit exists or not, and what it may be.

Remember though, these are just approximations!
\begin{defn}{Numerically}
  We can definitely see in the example above that even when we were pretty sure that the limit existed, it may be hard to tell exactly what it is.
  It could also be possible that the limit doesn't exist, because the left-sided limit and the right-sided limit are very similar but not actually the same.\\

  For instance, what if $\dlim_{x\to a^-} f(x) = 1.98325678$ and $\dlim_{x\to a^+} f(x) = 1.98325635$.
  It would be very hard to notice this difference numerically.
  We might (wrongly) conclude that the left-sided limit and right-sided limits are both $1.983256$\footnote{Notice how quickly we gave up on the numerical approximations for the examples above before we just concluded that the one-sided limits probably matched up.}, and then (mistakenly) say that $\dlim_{x\to a} f(x)=1.983256$ instead of saying it doesn't exist.
\end{defn}

\begin{defn}{Graphically}
  Imagine a similar situation as above, but we only get to look at the picture of the graph.
  If $\dlim_{x\to a^-} f(x) = 1.98325678$ and $\dlim_{x\to a^+} f(x) = 1.98325635$, we would likely not notice.
  On any normal scale, the difference between the $y$-value $1.98325678$ and the $y$-value $1.98325635$ would be imperceptible.\\

  Also, even if we're pretty sure that a limit might exist, if that limit doesn't fall on a ``nice'' value (like an integer, or some other value that matches up with the $y$-axis scale), then we may not be able to tell what it is.\footnote{Here's an example:
  % graph
  }
\end{defn}

We must approach these approximations with a grain of salt -- these approximations aren't perfect.
They are, however, very helpful.
We can gain valuable insight and build intuition about limits through these approximations.
Putting both of these approximation methods together for the same problem gives us a well-rounded view of the limit we're investigating, and a good amount of evidence for either existence or non-existence.

We'll practice this in these next few examples.

\section*{Examples}


\begin{enumerate}
  \item Let's approximate $\dlim_{x\to 2} \dfrac{x-4}{\sqrt{x}-2}$.

  \textbf{Left-Sided Limit}

  \begin{tabular}{ccccc} \toprule
    $\bm{x}$ & $1.5$ & $1.9$ & $1.99$ & $1.999$ \\ \midrule
    $\bm{f(x)}$ & $3.2247449$ & $3.3784049$ & $3.4106736$ & $3.4138600$\\ \bottomrule
  \end{tabular}

  \begin{flushright}
    \textbf{Right-Sided Limit}

    \begin{tabular}{ccccc} \toprule
      $2.001$ & $2.01$ & $2.1$ & $2.5$ & $\bm{x}$ \\ \midrule
      $3.4145671$ & $3.4177447$ & $3.4491377$ & $3.5811388$ & $\bm{f(x)}$ \\ \bottomrule
    \end{tabular}
  \end{flushright}

  Hopefully we're seeing the problem with approximations: it is not clear what the exact value of the $y$-value that we're getting close to is.
  It looks like it's approximately 3.41-ish, but that's not great.
  We'd like to get an exact value here!\footnote{We'll come back to this problem a little later when we're able to evaluate limits analytically.}

  You also might notice that this is the first example that isn't ``obvious:'' in all of the other examples, we could have just evaluated the function at the $x$-value we were considering.
  Here, though, if $f(x) = \dfrac{x-4}{\sqrt{x}-2}$, we can see that $f(2)$ is not defined.
  Strangely, though, it does look like the limit should exist -- the left-sided limit and the right-side limit look to be the same, although we can't easily tell what they are exactly.

  This example not only points out the problems with approximations (since again, we can't easily tell what the limit is exactly), but also reminds us that the limit of a function and the function value itself are not always related.
  \item Let's approximate $\dlim_{x\to 1} \dfrac{|x-1|}{x-1}$.
\end{enumerate}

Examine $\dlim_{x\to 0} \cos\left(\dfrac{1}{x}\right)$.
Numerically, it looks like $\dlim_{x\to 0} \cos\left(\dfrac{1}{x}\right)=-1$, but it actually oscillates between $-1$ and 1.

Note that when $x=\dfrac{1}{n\pi}$, we can see that  $\cos\left(\dfrac{1}{x}\right)=\cos(n\pi)$.
For bigger values of $n$, we know that $x=\dfrac{1}{n\pi} \to 0$, and we can also use some basic trig knowledge to analyze what happens when $n$ is odd and when $n$ is even.
So as $n$ gets bigger, it bounces back and forth between even and odd values, meaning the value of $\cos(n\pi)$ oscillates between $-1$ and $1$.

So, show that $\dlim_{x\to0^+}\cos(1/x)$ does not exist to show that the two-sided limit
