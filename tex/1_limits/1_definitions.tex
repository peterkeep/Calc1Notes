\section{Definitions of Limits}

We begin with the definition of the limit of a function.

\begin{thm}{Limit of a Function}
  Consider the function $f$, defined at all $x$ around some value $a$ except possibly at $a$ itself.
  If $f(x)$ is arbitrarily close to the single real number $L$ whenever $x$ is sufficiently close to, but not equal to, $a$, then we say that the limit of $f$ is $L$ as $x$ approaches $a$, and we write
  \[\lim_{x\to a} f(x) = L\]
  Alternatively, we can write that $f(x)\to L$ when $x\to a$.
\end{thm}

We note that in this definition, we do not actually consider $f(a)$.
In fact, the function value at $x=a$ doesn't even have to exist.
We only care about the function values \textit{around} the point at $x=a$.

When we investigate limits, we'll do so in three different ways\footnote{
These three methods of investigation will remain consistent for most new concepts we discuss in this text.}:
\begin{enumerate}
  \item Numerically
  \item Graphically
  \item Analytically
\end{enumerate}

In this section, we will discuss some numerical and graphical approximations of limits, and will build up to analytical evaluations of a limit in the next few sections.

\section*{Numerical Approximations}

Consider the function $f(x) = x-4$.

Let's say we want to investigate the behavior of the function \textit{around} $x=2$.
What are the function values doing around that point?
When $x$ is sufficiently close to $2$, is $f(x)$ arbitrarily close to some single real number?

If I were to ask you these questions, how would you approach answering them?

Let's start by just picking some $x$-values that are ``close to'' $2$.
What $x$-values are you thinking of?

Let's start with $x=1.5$.
We can find $f(1.5)$ pretty easily:
\begin{align*}
  f(1.5) & = 1.5-4\\
  & = -2.5
\end{align*}

Our predicament now is that we are trying to see what $y$-value (if any), this is arbitrarily close to.
We don't really hanve enough evidence to say right now.
Maybe we need an $x$-value that is \textit{closer} to $2$.

Let's try $x=1.9$.
\begin{align*}
  f(1.9) & = 1.9-4\\
  & = -2.1
\end{align*}

We might have an idea now of what $f(x)$ is close to.
We can always try this again, maybe with $x=1.99$ now.
\begin{align*}
  f(1.99) & = 1.99-4\\
  & = -2.01
\end{align*}

We can keep this going a bit more:

\begin{center}
  \begin{tabular}{cccccc} \toprule
    $\bm{x}$ & 1.5 & 1.9 & 1.99 & 1.999 & 1.9999\\ \midrule
    $\bm{f(x)}$ & $-2.5$ & $-2.1$ & $-2.01$ & $-2.001$ & $-2.0001$\\ \bottomrule
  \end{tabular}
\end{center}

So, do we have a good idea of what real number $f(x)$ is getting arbitrarily close to when $x$ is sufficiently close to $2$?

It should be obvious in this case: it looks like $f(x) \to -2$ when $x\to 2$.
We have some good evidence that $\dlim_{x\to 2}x-4 = -2$.

But wait, is this ok? We've tried to look at $x$-values that were sufficiently close to $2$, but we've only looked at $x$-values that are \textit{less than} 2.
Why not repeat this process with $x$-values that are close to, but slightly larger than, 2?

\begin{center}
  \begin{tabular}{cccccc} \toprule
    $\bm{x}$ & 2.5 & 2.1 & 2.01 & 2.001 & 2.0001\\ \midrule
    $\bm{f(x)}$ & $-1.5$ & $-1.9$ & $-1.99$ & $-1.999$ & $-1.9999$\\ \bottomrule
  \end{tabular}
\end{center}

Well this is good news! It still looks like $f(x) \to -2$ when $x\to 2$.
We probably have enough evidence to believe that $\dlim_{x\to 2}x-4 = -2$.

Maybe more importantly, though, we've touched on a very important concept with limits.

\subsection*{One-Sided Limits}

We've seen already, then, that there are two ``sides'' to a limit, since we can look at values of $x$ that are \textit{close to} $a$ on either side: $x$-values that less than $a$, and $x$-values that are greater than $a$.

\begin{defn}{Right-Sided Limit}
  If $f$ is arbitrarily close to the single real-number $L$ whenever $x$ is sufficiently close to, but greater than, $a$, then we write:
  \[ \lim_{x\to a^+} f(x)=L\]
\end{defn}

\begin{defn}{Left-Sided Limit}
  If $f$ is arbitrarily close to the single real-number $L$ whenever $x$ is sufficiently close to, but less than, $a$, then we write:
  \[ \lim_{x\to a^-} f(x)=L\]
\end{defn}

In our example above, whether we looked at $x$-values that were slightly less than $2$ or $x$-values that were slightly greater than $2$, the function values seemed to be getting arbitrarily close to $-2$.
Since both the right-sided limit and the left-sided limit were the same, we decided that we had more than enough evidence to conclude that the limit of the function as $x$ approached $2$ was $-2$.
What would we have concluded if the one-sided limits did not match?

\begin{thm}{The Existence (and Non-Existence) of a Limit}
  If $f$ is some function defined at all $x$-values on either side of $a$ except possibly at $a$ itself and $L$ is some real number, then we say that $\dlim_{x\to a} f(x) = L$ if and only if $\dlim_{x\to a^-}f(x) = L$ and $\dlim_{x\to a^+} f(x) = L$.

  If $\dlim_{x\to a^-} f(x) \neq \dlim_{x\to a^+} f(x)$, then we say that $\dlim_{x\to a} f(x)$ does not exist.
\end{thm}

\section*{Graphical Approximations}



\section*{Examples}

\begin{itemize}
  \item Graph a small piecewise function with a hole, but the limit existing.
  \item Graph a small piecewise function where the left side limit does not equal the right sided limit.
  \item Numerically find:
    \begin{itemize}
      \item $\dlim_{x\to 4} \dfrac{x-4}{\sqrt{x}-2}$
      \item $\dlim_{x\to 1} \dfrac{\sqrt{x}-1}{x-1}$
      \item $\dlim_{x\to 2} \dfrac{x^3-8}{4(x-2)}$
    \end{itemize}
  \item Examine $\dlim_{x\to 0} \cos\left(\dfrac{1}{x}\right)$.
  Numerically, it looks like $\dlim_{x\to 0} \cos\left(\dfrac{1}{x}\right)=-1$, but it actually oscillates between $-1$ and 1.

  Note that when $x=\dfrac{1}{n\pi}$, we can see that  $\cos\left(\dfrac{1}{x}\right)=\cos(n\pi)$.
  For bigger values of $n$, we know that $x=\dfrac{1}{n\pi} \to 0$, and we can also use some basic trig knowledge to analyze what happens when $n$ is odd and when $n$ is even.
  So as $n$ gets bigger, it bounces back and forth between even and odd values, meaning the value of $\cos(n\pi)$ oscillates between $-1$ and $1$.

  So, show that $\dlim_{x\to0^+}\cos(1/x)$ does not exist to show that the two-sided limit does not exist.
\end{itemize}

