\section{Properties of Limits}

We've estimated limits numerically and graphically, but these can be difficult without more tools at our disposal. Here we'll build up some analytic methods of evaluating limits.

\begin{thm}{Limit Laws and Properties}
Let $f$ and $g$ be two functions, and assume that $\dlim_{x\to a} f(x)$ and $\dlim_{x\to a} g(x)$ both exist. Then:
\begin{enumerate}
\item $\dlim_{x\to a} k = k$ for any constant $k$
\item $\dlim_{x\to a} x = a$
\item $\dlim_{x\to a} [f(x)\pm g(x)] = \lim_{x\to a} f(x) \pm \lim_{x\to a} g(x)$
\item $\displaystyle\lim_{x\to a} cf(x) = c\lim_{x\to a} f(x)$ for any constant multiple (coefficient) $c$. This is really just repeatedly adding a function to itself.
\item $\dlim_{x\to a} [f(x)\cdot g(x)] = \left[\lim_{x\to a} f(x)\right] \left[\lim_{x\to a} g(x)\right]$
\item $\dlim_{x\to a} \frac{f(x)}{g(x)} = \frac{\dlim_{x\to a} f(x)}{\dlim_{x\to a} g(x)}$ if $\dlim_{x\to a} g(x)\neq0$
\item $\dlim_{x\to a} (f(x))^n = \left(\dlim_{x\to a} f(x) \right)^n$ -- exponents are really just repeated multiplication.
\item $\dlim_{x\to a} (f(x))^{n/m} = \left(\lim_{x\to a} f(x)\right)^{n/m} = \sqrt[m]{\left(\lim_{x\to a}f(x)\right)^n}$ NOTE: if $m$ is even, then we need $f(x)\geq 0$ when $x$ is close to $a$.
\end{enumerate}
\end{thm}

\begin{note}{Examples}
Given $\dlim_{x\to 1} f(x) = 4$ and $\dlim_{x\to 1} g(x) = 2$, evaluate the following limits:
\begin{enumerate}
\item $\dlim_{x\to 1}\dfrac{f(x)-g(x)}{f(x)}$
\item $\dlim_{x\to 1} \sqrt{g(x)^2+f(x)^2}$
\item $\dlim_{x\to 1} 3f(x)+\dfrac{2g(x)}{5}$
\end{enumerate}
\end{note}
