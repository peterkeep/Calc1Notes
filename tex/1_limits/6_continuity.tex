\section{Continuity}

Earlier, we classified polynomial and rational functions as ``easy'' limits (we could just check the function outputs for the limits).
We'll expand that definition of ``easy'' limits here with a more formal definition of continuity.

What does continuity mean? What does it look like? What does it look like to ``break'' continuity?

\begin{defn}{Continuity at a Point}
  A function is continuous at a point $a$ if $\dlim_{x\to a}f(x) = f(a)$.
  If this equality doesn't hold, we'll call $a$ a \textbf{point of discontinuity}.
\end{defn}

\begin{note}{Note}
  The the continuity limit equality to hold, we need three things to be true:
  \begin{enumerate}
    \item $f(a)$ is defined ($a$ is in the domain of $f$)
    \item $\dlim_{x\to a}f(x)$ exists ($\dlim_{x\to a^-} f(x) = \dlim_{x\to a^+} f(x)$)
    \item $\dlim_{x\to a}f(x) = f(a)$ (limit and function value are the same).
  \end{enumerate}
\end{note}

This definition is a lot more robust than it seems.
Using limits to define continuity adds a lot of depth to a concept that we don't normally consider that interesting.

\textit{Have students design functions that are either weird looking but continuous or break the definition of continuity somewhere.}

\begin{thm}{Properties of Continuity}
  If two functions $f$ and $g$ are both continuous at $a$, then the following functions are also continuous at $a$:
    \begin{enumerate}
      \item $f+g$
      \item $f-g$
      \item $fg$
      \item $f/g$ as long as $g(a)\neq 0$
      \item $cf$ for some coefficient $c$
      \item $(f(x))^n$ for $n>0$
    \end{enumerate}
\end{thm}

This is relatively easy to prove once we know that limits are preserved across all of these operations.

\begin{note}{Note}
  Polynomial functions are continuous for all $x$, and rational functions are continuous wherever they don't divide by 0.
\end{note}

Also, if $g$ is continuous at $a$ and $f$ is continuous at $g(a)$, the function $f\circ g$ is continuous at $a$.

For these composite functions to be continuous, we should at least talk about how limits act with composite functions in general.

\begin{thm}{Limits of Composite Functions}\hspace{1cm}
  \begin{enumerate}
    \item If $g$ is continuous at $a$ and $f$ is continuous at $g(a)$ then $\dlim_{x\to a} f(g(x)) = f\left(\dlim_{x\to a} g(x)\right)$.
    \item Even if $g$ is not continuous if $\dlim_{x\to a} g(x)=L$ and $f$ is continuous at $L$, then $\dlim_{x\to a}f(g(x))=f\left( \dlim_{x\to a} g(x)\right)$.
  \end{enumerate}
\end{thm}

\begin{note}{Examples}
  Evaluate the following limits of composite functions:
  \begin{multicols}{3}
    $\dlim_{x\to0} \left(\dfrac{x^4-6x^2+2}{x^6+3x^4+1}\right)^8$

    $\dlim_{x\to3} \sin\left(\dfrac{x^2-9}{x-3}\right)$

    $\dlim_{x\to3} \sqrt{x^2-5}$
  \end{multicols}
  Notice that in these examples, we have composition of functions where the ``inside'' functions are rational or polynomial, and so are continuous (wherever the denominator is not 0). So the insides are ``easy'' limits.
  The ``outside'' functions are $x^8$, $\sin(x)$, and $\sqrt{x}$ respectively.
  We know the first one is continuous, we're probably pretty sure that $\sin(x)$ is continuous (based on what we knew about continuity before today), but the $\sqrt{x}$ function is a bit different.
  We've been using this composition rule for the square root function, but we should make sure things are looking ok still.
\end{note}

\subsection*{Continuity on an Interval}

We'll focus on functions being continuous not just at a single, carefully picked point, but on a whole interval.
This is a weird problem with the square root function:

Is $f(x) = \sqrt{x}$ continuous at $x=0$?

\begin{defn}{Coninuity at Endpoints}
  Sometimes this is called one-sided continuity.
  A function $f$ is continuous from the left (or left-continuous) at $a$ if $\dlim_{x\to a^-}f(x) = f(a)$.
  It is continuous from the right (or right-continuous) at $a$ if $\dlim_{x\to a^+}f(x) = f(a)$.
\end{defn}

This assures us that even thought $\dlim_{x\to 0}\sqrt{x}$ does not exist (since the left side does not exist), we can still talk about $\sqrt{x}$ being continuous from the right at that point.

\begin{defn}{Continuity on an Interval}
  A function $f$ is continuous on the interval $I$ if for every point $c\in I$, $f$ is continuous at $c$.
  If $I$ includes its endpoints, then $I$ just needs to be continuous from the left or right (whichever is appropriate).
\end{defn}

\textit{Draw some piecewise functions to talk about continuity and intervals of continuity.}

What if we extend this from just intervals to sets of numbers?
Can a function be continuous on $\mathbb{Z}$? $\mathbb{Q}$? Its domain?

\textit{Show off the Interesting Example.}

\subsection*{Let's Circle Back to Functions with Roots}

We know that limits work out pretty easily for the most part with limits, as long as we're careful with negatives and even roots.
So we know $\dlim_{x\to a} (f(x))^{n/m} = \left(\dlim_{x\to a}f(x)\right)^{n/m}$, with the restriction that $f(x)\geq 0$ if $m$ is even (and $n/m$ is in lowest terms).
So as long as $m$ is odd, we don't have to worry about anything: if $f(x)$ is continuous at $a$, then the root is too.

If $m$ is even, then we definitely need to make sure that $f(x)\geq 0$.

\begin{thm}{Statement/Claim}
  If $f$ is continuous at $a$ and $f(a)>0$, then $f(x)>0$ for all values $x$ in the domain of $f$ in some interval around $a$.
\end{thm}

\begin{note}{Note}
  We're going to be super petty about how we pick the interval.
  Just keep making the interval around $a$ smaller and smaller.
\end{note}

So all of this to say:

\begin{thm}{Continuity of Functions with Roots}
  Assume that $m$ and $n$ are positive integers with no common factors (so $n/m$ is reduced).
  \begin{itemize}
    \item If $m$ is odd, then $(f(x))^{n/m}$ is continuous whenever $f$ is continuous.
    \item If $m$ is even, then $(f(x))^{n/m}$ is continuous whenever $f$ is continuous and $f(x)>0$.
  \end{itemize}
\end{thm}

\begin{note}{Examples}
  When are the following functions continuous?

  $f(x)=\sqrt{16-x^2}$ \quad $g(x) = \sqrt{x^3-8}$ \quad $h(x) = (x^2+2x-8)^{2/3}$
\end{note}

\subsection*{Transcendental Functions}

\begin{defn}{Trigonometric Functions}
  We've used the squeeze theorem on a couple of trigonometric functions to show that some specific limits were the same as the function values at those points.

  All trigonometric functions ($y=\sin x, y=\cos x, y=\tan x, y=\sec x, y=\csc x, y=\cot x$) are all continuous on their domains.
  \begin{note}{Note}
    We still get screwed up at asymptotes, but the functions aren't defined there anyways.
  \end{note}
\end{defn}

\begin{defn}{Exponential Functions}
  It feels like any exponential function defined in the form $f(x)=b^x$ ($b>0, b\neq1$) should be continuous, but we need to formalize the function itself.
  We know what $b^x$ means for $x\in\mathbb{Q}$, but what about irrational exponents? What does $4^{\sqrt{2}}$ mean?

  For now, we'll just say that it's continuous on its domain without showing it.
\end{defn}

\subsection*{Intermediate Value Theorem}

This is used a lot to make sure that some solution to an equation $f(x)=L$ actually exists.
Sometimes we're trying to make sure it exists before we find it, and other times we just care about its existence.

\textit{Illustrate with crossing the room. Example of people leaving during a fire drill.}

\begin{thm}{The Intermediate Value Theorem}
  Suppose $f$ is continuous on the interval $[a,b]$ and $L$ is some real number such that $f(a)<L<f(b)$ or $f(b)<L<f(a)$.
  Then there exists at least one number $c\in(a,b)$ such that $f(c)=L$.
\end{thm}
